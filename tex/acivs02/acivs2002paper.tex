%This document provides instructions for acivs 2002 authors
%You can also use it as a template for writing your paper

\documentclass{article}
\usepackage{acivs2002,amssymb,amsmath,amsfonts,epsfig}
\usepackage[english]{babel}
\setcounter{page}{1}

%\ninept

\title{On the destruction of modern computing technologies}

%the names of all the authors
%uncomment the following line for muliple authors
\name{$^1$Homer Simpson, $^2$Evil Bert, $^3$Telly Tubby and $^4$Bill Gates}

%uncomment the following line for one author
%\name{Homer Simpson}


%the affiliations of all the authors
%%uncomment the following lines for multiple authors
\address{
%email of first author
$^1${\tt Homer.Simpson@coldmail.moc}\\
%affiliation and address of first author
$^1$Springfield University, Dept. of Cartoon Engineering, 12 Banana
Road, Springfield, USA\\
%affiliations of remaining authors 
$^2$Institute for World Domination, 666 Evil Road, Cartoonland\\
$^3$Microsoft Corporation, 1 Microsoft Drive, Seattle
}


%uncomment the following lines for a single author
%\address{
%%email of first author
%{\tt Homer.Simpson@coldmail.moc}\\
%%affiliation and address of only author
%Springfield University, Dept. of Cartoon Engineering, 12 Banana
%Road, Springfield, USA\\
%}




\begin{document}
\maketitle

\begin{abstract}
{\ninept
This is the template file for the CD-rom proceedings of ACIVS 2000. It
was kindly provided by the organisers of the 2nd IEEE Benelux Signal
Processing Symposium. This template has been generated from {\sc WASPAA} 99 
and {\sc ICASSP} 2000 templates and aims at producing conference proceedings
in electronic form. The format is essentially the one used for ICASSP
conferences. 
Please use either this LaTeX or the accompanying Word97 formats when preparing your submission. All questions concerning the submission should be addressed to philips@telin.rug.ac.be.
The templates are available in electronic form at the conference website:\\ http://telin.rug.ac.be/acivs2000/.
}
\end{abstract}

\section{Introduction}

Please follow these  instructions:
\begin{itemize}
\item All manuscripts must be in English.
\item All papers will be published in Adobe's pdf-format on the conference proceedings cd-rom.
\item The paper should be no larger than 5 Mbyte after its final conversion to pdf format. Conversion to pdf usually reduces file size. Therefore the source document (including external files, e.g., those containing linked images) may be larger. We allow a maximum of 20 Mbyte of source files. However, we strongly recommend using smaller file sizes to reduce the risk of problems in browsing the CD-rom.
\item Papers should be submitted in two formats:
  \begin{enumerate}
  \item publication and review format: a single pdf- or postscript
    file for reviewing purposes and for publication purposes (if it is
    of sufficient quality). Please carefully follow the instructions
    below concerning generation of ps- and pdf-documents. This is
    necessary to ensure an optimal readability. 
  \item source format: the document source files, i.e., a word
    document with embedded figures or a LaTeX document with
    accompanying postscript figures. We will use these files only if
    the ps- or pdf-file you provide proves inadequate for publication
    purposes.  
    \par
    If multiple source files are submitted, they should be placed
    together in one directory and this directory should be archived
    into a single file (e.g., using ``zip'', ``tar -zcf''). Acceptable
    archive formats are ``.tar.gz'', ``.zip'', ``.tar.z'',
    ``.tar.bz2''.\\
    Please check that all necessary files are present. For
    instance, if you use LaTeX do not forget to include any
    non-standard style-file that you have used.  
    
  \end{enumerate}
\end{itemize}

\section{Style Guidelines}
\begin{itemize}
\item If you use LaTeX or MSWord please use the templates for
  preparing your paper. This will ensure a uniform look of the
  proceedings. In any case, please adhere to the following style
  guidelines. 
\item The paper must be {\em 4-8 pages in length.}  
\item To achieve the best viewing experience we strongly encourage to
  use Times-Roman font (the LaTeX style file as well as the Word
  template files use Times-Roman). {\em If a font face is used that is not
  recognized by the submission system, your proposal will not be
  reproduced correctly.}
\item The paper must be in the following format: {\em A4 format,} single
  spaced, two (2) columns, printed or typed in black ink, no smaller
  than nine (9) point type font throughout the paper, including figure
  captions. In fact, we strongly recommend that you use ten (10) point
  type fonts for the main text in the paper. In the abstract and the
  references, you can use nine (9) point type fonts. 
\item Please turn on hyphenation (if you don't the result may be ugly
  because of the narrow column width). 
\item Use italic typeface to emphasize words; {\em never underline titles
  or word that need to be emphasized.} 
\item MSWord users: please note that superfluous space characters will
  interfere with MSWord's text justification. Please use "find and
  replace" to replace multiple spaces by a single space.  
\item All text and figures must be contained in a 178 mm x 226 mm (6.9
  inch x 8.9 inch) image area.  
\item The left margin must be 19 mm (0.75 inch). The top margin must
  be 25 mm (1.0 inch). 
\item  Text should appear in two columns, each 85 mm wide with 8 mm
  space between columns.  
\end{itemize}

\section{Paper title}
The paper title has to appear in capital letters, boldface if
possible, centered across the top of the two columns on the first page
as indicated above. The authors' name(s) and affiliation(s) appear
below the title in capital and lower case letters. If space permits
include a mailing address here. The template in the Authors' Kit
indicates the image area where the title and author information should
go. These items need not be strictly confined to the number of lines
indicated; papers with multiple authors and affiliations, for example,
may require two or more lines for this information. However, the
placement of the title should be immediately below the top line. 

If there is only one author, please remove the numerical superscript
 in front of the author name and affiliation.  

\subsection{Abstract}
Each paper should contain an abstract of about 100-200 words that appears at the beginning of the paper.

\subsection{Figures}
All figures should be centered on the column (or page, if the figure
spans both columns). Figure captions should follow each figure and
have the format given in the example. If you use colors in
illustrations please verify that legibility is not lost when the paper
is printed in grey-scale. Position figures on top of the page. 

The easiest way to position figures and captions in MSWord (read: "the
least difficult way") is to use the "square wrapping style." Turn off
"move object with text" and turn on "lock anchor," after making sure
that the anchors of the figure and of its caption are located in
exactly the same position.  

\begin{figure}[t]
\centerline{\epsfig{figure=figure.eps,width=75mm}}
\caption{{\it Directivity measurement of a trumpet.}}  
\label{fig:figure1}
\end{figure}

In LaTeX please use the following construction to include an eps figure:
\begin{verbatim}
\begin{figure}[t]
\centerline{
\epsfig{figure=figure.eps,width=75mm}}
\caption{{\it Directivity measurement 
of a trumpet.}}  
\label{fig:figure1}
\end{figure}
\end{verbatim}

\subsection{Equations}
Equations should be placed on separate lines and numbered: 
\begin{equation}
{\bf W}_{WF} = X^{-T} \cdot \textrm{diag} \{ \frac{\sigma_i^2-\eta_i^2}{\sigma_i^2} \} \cdot X^T.
\label{eq:equation1}
\end{equation}

\subsection{Page Numbers}
Page numbers will be modified electronically, so please leave the
numbering as is, i.e., the first page will start at page S00-1 and the
last page will be, e.g., S00-4. 

\subsection{References}
List and number all references at the end of the paper. The references
can be numbered in alphabetic order or in order of appearance in the
document. When you add a new reference, insert a bookmark between the
number and the first word of the reference. You will then be able to
refer to the reference in the text by inserting a cross-reference to
the bookmark. This approach guarantees that the numbering will be
correct in the printed document.  

In MSWord use "tools, options, view, bookmarks" to view the location
of bookmarks. Note that numbers are only updated when you print (or
print preview) the document.

In any case, use reference numbers in square brackets as shown at the
end of this sentence \cite{sps1} \cite{sps3}. The reference format is
the standard IEEE one \cite{sps2}. Check reference numbering before
you submit your paper.  

\section{Headings}
Major headings appear in capital letters, bold face if possible,
centered in the column. 

\subsection{Sub Headings}
Sub headings appear in capital and lower case, either underlined or in
boldface. They start at the left margin on a separate line. 

\subsubsection{Sub-Sub Headings}
Sub-sub headings appear in capital and lower case, indented like a
paragraph and on a separate line and in italic typeface. 

\section{Pdf- and postscript generation}
Please consult the instructions on the website concering pdf- and
ps-generation. It is absolutely essential that you follow these
guidelines, because most text processing programs (LaTeX and MSWord)
will by default yield ugly looking papers (e.g., "print to file",
"dvips", "create pdf", ...). 

\bibliographystyle{IEEEbib}
\bibliography{acivs2002paper}

\end{document}
