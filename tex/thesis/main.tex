


%%%%%%%%%%%%%%%%%%%%%%%%%%%%%%%%%%%%%%%%%%%%%%%%%%%%%%%%%%%%%%%

\chapter{Introduction}

%%%%%%%%%%%%%%%%%%%%%%%%%%%%%%%%%%%%%%%%%%%%%%%%%%%%%%%%%%%%%%%



\section{Automated reading} %---------------------------------

We are interested in methods which enable a computer to read text in a similar environment to humans.  Applications which would serve useful in these situations are like wearable helpers for seeing or blind people.

Blah blah.



\section{Focus of research} %---------------------------------

The problem is defined mainly by the requirements of a number of potential applications and the expected state of available hardware in the near future.

We overlook the use of sequences/motion.  Justify.

Mention the social benefits to the blind.

Mention some statistics on the forms of text being distributed in society.
  (Literature, leaflets, business post, magazines, advertising, notices/signs, product labelling.)
	What proportion of these are black and white?  On flat surfaces?



\section{Deconstruction of the problem} %----------------------

Locating likely regions, assesment, orientation retrieval, recovery, OCR.

OCR straight from image.

Justification for doing OCR separately: software already exists, humans prefer to read text face-on.

Psychology:  Things humans use to locate and read text:
	- Context.  They seek text in places it is expected to appear.  (includes "flat surfaces")
	- Visual appearance:  Generally two colour / high contrast, texture: lines of squiggles.
	- Tendency to seek text then move closer for further inspection.
	- Eye has low resolution for colour, texture, general scene, and high resolution centre for inspection of candidates.  (Does this support our method or imply that comparison is less applicable?)

Features available: texture and colour, binary properties, local features.



\section{Integration versus independence} %--------------------

% NOTE: Difficult to discuss "routes" without first briefly mentioning possibile starting points / sources.

Structure of system.

Possibilities for information flow.  Different paths to solution.  What do we solve first?

How sequential should the process be?  Ambiguity should be passed forwards.

Benefit of observing as many different things as possible about the image, and feeding them through a high-level pattern recognition system.



\section{Overview of this thesis} %-----------------------------



%%%%%%%%%%%%%%%%%%%%%%%%%%%%%%%%%%%%%%%%%%%%%%%%%%%%%%%%%%%%%%%

\chapter{Locating text in the image}

%%%%%%%%%%%%%%%%%%%%%%%%%%%%%%%%%%%%%%%%%%%%%%%%%%%%%%%%%%%%%%%

\section{Approaches} %------------------------------

The need to generate multiple hypothesese.  Can this be applied to the features as well as the classifier in the texture method?

Search algorithm:

Examine small windows or whole image?

The cluster, assess, reject/process, then continue with rest method.

Top-down optimised classification algorithm.



\section{Quads approach using local features} %-----------------

(Everything up until now has been theory.  Now we present some results of practical experiments.)



\section{Texture classification approach} %---------------------

Overview of existing general texture techniques.

Discussion of binary topological analysis.  (Eg. no islands within holes)

Our simple classification NN and our features adapted from OCR research are implemented.

Top-down multiresolution method not yet implemented, but partway there.

The cluster, assess, reject method is not implemented.



\section{Quads and Texture} %------------------------------------

Example of combining the methods, or at least just discussion.



%%%%%%%%%%%%%%%%%%%%%%%%%%%%%%%%%%%%%%%%%%%%%%%%%%%%%%%%%%%%%%%

\chapter{Determining text orientation}

%%%%%%%%%%%%%%%%%%%%%%%%%%%%%%%%%%%%%%%%%%%%%%%%%%%%%%%%%%%%%%%



From texture, ie. frequency analysis (a la Zissermann...).

From structured model matching (our paragraph recognition).



% \section{Text orientation by paragraph recognition} %---------
\section{Locating horizontal vanishing point of paragraph} %---------

\section{Locating vertical vanishing point using paragraph edges} %---------

\section{Locating vertical vanishing point using line spacing} %---------




%%%%%%%%%%%%%%%%%%%%%%%%%%%%%%%%%%%%%%%%%%%%%%%%%%%%%%%%%%%%%%%

\chapter{Recovery, post-processing, and results}

%%%%%%%%%%%%%%%%%%%%%%%%%%%%%%%%%%%%%%%%%%%%%%%%%%%%%%%%%%%%%%%

Recovery of focal length.

Discussion of correct pixel $\rightarrow$  pixel mapping.

Post processing before feeding to OCR.




%%%%%%%%%%%%%%%%%%%%%%%%%%%%%%%%%%%%%%%%%%%%%%%%%%%%%%%%%%%%%%%

\chapter{Results}

%%%%%%%%%%%%%%%%%%%%%%%%%%%%%%%%%%%%%%%%%%%%%%%%%%%%%%%%%%%%%%%

\%age paragraphs/documents correctly recovered in ideal scenes.

\%age paragraphs/documents correctly recovered in random scenes.

Comparison of OCR results from our recovery agains scanned images.

Using paragraph recognition to support quadrilateral hypothesese.
eg. \%age correct quadrilateral hypothesese supported by paragraph.



%%%%%%%%%%%%%%%%%%%%%%%%%%%%%%%%%%%%%%%%%%%%%%%%%%%%%%%%%%%%%%%

\chapter{Conclusions}

%%%%%%%%%%%%%%%%%%%%%%%%%%%%%%%%%%%%%%%%%%%%%%%%%%%%%%%%%%%%%%%



\section{Summary and discussion} %-----------------------------



\section{Future work} %----------------------------------------

Should we try to present all of our past research in this document?  Or should we build on it to develop a better system?

How much theory versus practice?  Presumably a wider area of theory can be covered than is covered by the experiments.  (Ie. not {\em all} theory presented need be backed up by practical experiments.



