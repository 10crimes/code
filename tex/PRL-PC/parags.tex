

\section{Determining the Style of Paragraph Formatting} \label{sec-parags}

The location of the horizontal vanishing point, and the projection profile of
the text from that position, make it possible to separate the individual lines
of text.  This will allow the style of formatting or justification of the
paragraph to be determined, and lead to the location of the vertical vanishing
point as shall be described in \refsect{sec-vertvanish}.

We apply a simple algorithm to the winning projection profile to segment the
lines.  A {\em peak} is defined to be any range of angles over which all the
projection profile's bins register more than $K$ pixels, taken as the average
height of the interesting part of the projection profile:

\begin{equation}
K= \frac{1}{y-x+1} \sum_{i=x}^{y}B_i
\end{equation}

{\parindent 0mm
where $x$ and $y$ are the indices of the first and last non-empty bins respectively.
A {\em trough} is defined to be the range of angles between one peak and the next.  
The central angle of each trough is used to indicate the separating boundary of
two adjacent lines in the paragraph.  We project segmenting lines from the
vanishing point through each of these central angles.
% in the range.
All pixels in the binary image lying between two adjacent segmenting
lines are collected together as one line of text. Example results of this
segmentation are shown in \reffig{linesegfig}.  Most lines of
text are segmented accurately, although in \reffig{chem002overlay} a very short
line has been ignored.  Noisy pixels, very short lines, and extraneous document
elements may become attached to a true text line, or be segmented as a separate
line.  However, the processing which follows will compensate for this irrelevant
data. 
} 


\begin{figure}[t]
\centering
\begin{center}
% \subfigure[A fully formatted document]{\epsfig{figure=images/chem002overlay.ps,width=64mm}\label{chem002overlay}\label{summaryfiga}}
\subfigure[A left justified paragraph]{\epsfig{figure=images/prlrunning/origover.eps,width=64mm}\label{chem002overlay}\label{summaryfiga}}
\hspace{2mm}
% \subfigure[A centrally aligned document]{\epsfig{figure=images/prlall/chem017/origover.eps,width=64mm}\label{chem010overlay}}
\subfigure[Two centrally aligned paragraphs]{\epsfig{figure=images/prlall/chem015/overboth.eps,width=64mm}\label{chem010overlay}}
% \subfigure[]{\epsfig{figure=images/chem005001origover.ps,width=58mm}\label{chem005001overlay}}
\end{center}
\vspace*{0mm}
\caption{Example paragraphs: 
Line segmentations are marked in white, points for margin fitting in green
(used) and red (rejected outliers), the baseline in yellow;
the rectangular frame on the text plane in blue.}
\label{linesegfig}
\label{summaryfig}
\end{figure}



Depending on the formatting of the paragraph being recovered, there are now two
possible ways to analyse the segmented lines to reveal the vertical vanishing
point.  If the paragraph is {\em fully justified}, then the left and right
margins of the text are straight, and intersecting these two margin lines will
provide us with the vertical vanishing point, and the problem is fully resolved.
Alternatively, if the paragraph is {\em left justified}, {\em right justified},
or {\em centred},
a straight line will be visible either on the left margin, on the right margin,
or through the centres of the lines.  The vanishing point will lie somewhere
along this baseline.  However, the actual position of the vanishing point will
still be unknown, and must be estimated.


Initially, we must determine the structure of the paragraph, i.e. its 
formatting style. We collect the left end, the centroid, and the right end of each
of the segmented lines, to form three sets of points $P_L,P_C,P_R$ respectively.
Since some justification or formatting is anticipated in the paragraph, we will
expect a straight line to fit well through at least one of these sets of points,
representing the left or right margin, or the centre line of the paragraph.  To
establish the line of best fit for each set of points, we use a RANSAC (random
sampling concensus, \cite{bolles81ransac-based}) algorithm to reject outliers
caused for example by short lines, equations or headings.  Given a set of points
$P$, the line of best fit through a potential fit 
$F=\{\myvec{p}_i, i=1,..,M\}\subseteq P$ passes through $\myvec{c}$, the average
of the points, at an angle $\psi$ found by minimising the following error function:

\begin{equation}
% E_F(\myvec{c},\myvec{n}) = \frac{1}{M^5} \sum_{i=1}^{M} ( (\myvec{p}_i-\myvec{c}) \cdot \myvec{n})^2
E_F(\psi) = \frac{1}{M^5} \sum_{i=1}^{M} ( (\myvec{p}_i-\myvec{c}) \cdot \myvec{n})^2
\label{ransacerroreqn}
\end{equation}

{\parindent 0mm
where $\myvec{n}=(-\sin \psi,\cos \psi)$ is the normal to the line, $M^{2}$
normalises the sum, and a further $M^{3}$ rewards the fit for using a large
number of points. Hence, for the three sets of points $P_L,P_C,P_R$ we obtain
three lines of best fit $F_L,F_C,F_R$ with respective errors
$E_L,E_C,E_R$.  It is now possible to classify the formatting style of the
paragraph using the rules in \reftab{typeofparatable}.  \reffig{summaryfiga}
shows the line $F_L$ passing through the left margin of the paragraph.  In this
case $E_L<E_C$ and $E_L<E_R$, hence the second condition in
\reftab{typeofparatable} is satisfied and the paragraph is correctly identified
as being left justified.
}

\begin{table}[t]
  \begin{center}
    \begin{tabular}{|c|c|}
      \hline
      {\bf Condition} & {\bf Type of paragraph} \\
      \hline \hline
      % All three lines have low and similar error:
      $E_L \simeq E_C \simeq E_R$ & Fully justified. \\
      \hline
%      $E_L<E_R$ and $E_L<E_C$ & Left justified. \\
      $\min(E_L,E_C,E_R)=E_L$ & Left justified. \\
      \hline
%      $E_R<E_L$ and $E_R<E_C$ & Right justified. \\
      $\min(E_L,E_C,E_R)=E_R$ & Right justified. \\
      \hline
%      $E_C<E_L$ and $E_C<E_R$ & Centrally aligned. \\
      $\min(E_L,E_C,E_R)=E_C$ & Centrally justified. \\
      \hline
    \end{tabular}
  \end{center}
  % \vspace*{-5mm}
  \caption{Classifying the type of paragraph}
%  \vspace{-5mm} % This is a dirty hack to help page flow
  \label{typeofparatable}
\end{table}



For fully justified paragraphs, the recovery of the vertical vanishing point is
trivial, and may be achieved by intersecting the left and right margins of the
paragraphs, the results of which are shown later in \reftab{accuracytable} and
\reffig{vvpaccuracya}.  However, for a left justified, right justified or
centralised paragraph, we can retrieve only one baseline.  The other two fitted
lines will have significant errors due to the jagged margin(s).  In these
situations, a different method must be used to determine the position on the
baseline at which the vanishing point lies.





