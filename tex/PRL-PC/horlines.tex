\section{Locating the Horizontal Vanishing Point}  \label{locatehvpsect}

In \cite{messelod1}, Messelodi and Modena demonstrated a text location method on
a database of images of book covers.  They employed projection profiles to
estimate the skew angle (in the view plane) of the text.  A number of potential
angles were found from pairs of components in the text, and a projection profile
was generated for each angle.  They observed that the projection profile with
the minimum entropy corresponded to the correct skew angle.  This is to be
expected since the profile at the correct angle will have well-defined peaks and
troughs corresponding to each line of text and the gaps between them.
Projection profiles at other angles will cause lines to overlap, merging peaks
and troughs, and increasing the entropy of the profile.  This guided 1D search
is not directly applicable to our problem where the orientation of the text is
not parallel to the camera plane. Instead, the {\em vanishing point} in
$\mathbb{R}^2$, with two degrees of freedom, must be found. In order to search
this space, we will generate projection profiles from the point of view of
vanishing points, rather than from skew angles.



We need a circular search space $C$ as illustrated in \reffig{searchspacediag}.
% although a rectangular grid of polar coordinates would also work effectively.
Each cell $c=(r,\theta)$, $r\in[0,1)$ and $\theta\in[0,2\pi)$, in the space $C$
corresponds to a hypothesised vanishing point $\myvec{V} = (V_r,V_{\theta})$ on
the image plane $\mathbb{R}^2$, with scalar distance $V_{r}=r/{(1-r)}$ from the
centre of the image, and angle $V_{\theta}=\theta$.  This maps the infinite
plane $\mathbb{R}^2$ exponentially into the finite search space $C$.
A projection profile of the text is generated for
every vanishing point in $C$, except those lying within
the text region itself (the central hole in \reffig{ppmap}).


\begin{figure}[ht!]
\centering
\begin{center}
	\subfigure[Relationship between search space $C$ and $\mathbb{R}^2$]{
		\epsfig{figure=images/ppsearchspaceH.eps,width=45mm}
		\label{searchspacediag}
	}
	\hspace{5mm}
	% \subfigure[Scores for all projection profiles in $C$ generated from \reffig{runbin}]{\epsfig{figure=images/chem002ppmap.ps,width=45mm}\label{ppmap}}
	\subfigure[Scores for all projection profiles in $C$ generated from \reffig{runbin}]{\epsfig{figure=images/prlrunning/ppmap.eps,width=45mm}\label{ppmap}}
\end{center}
\vspace*{0mm}
\caption{Search space $C$}
\label{searchspace}
\end{figure}

%The projection profile  with respect to a vanishing point
%$\myvec{V}$ is obtained by collecting each pixel into a bin representing the
%angle between the vanishing point and the pixel.  Relative to the vanishing
%point, the text region will fall within a small range of angles, depending on
%the distance and position of the vanishing point.  In order to accumulate
%projection profiles which are comparable to each other, the angular range over
%which each projection profile is taken is set accordingly.  In other words, the
%left and right bins of the projection profile correspond to the left and right
%angles within which the text region lies.  This ensures that the projection
%profile of a distant vanishing point will not accumulate a tighter distribution,
%as would be the case if all projection profiles were accumulated in the range
%$0-2\pi$.
%%*** !Kill above paragraph?  Keep next but make clearer! ***
%%*** !Alternative paragraph to the last!  Which do you prefer?!

A projection profile $B$ is a set of bins $\{B_i, i=0,..,N\}$ into which image
pixels are accumulated.  In the classical 2D case, to generate the projection
profile of a binary image from a particular angle $\phi$, each positive pixel
$\myvec{p}$ is assigned to bin $B_i$, where $i$ is dependent on $\myvec{p}$ and
$\phi$ according to the following equation:

\begin{equation} \label{classicproj} 
i(\myvec{p},\phi) = \frac{1}{2} N + N
\frac{ \myvec{p} \cdot \myvec{U} }{ s } 
\end{equation}

{ \parindent 0mm where $\myvec{U}=(\sin \phi,\cos \phi)$ is a normal vector
describing the angle of the projection profile, and $s$ is the diagonal
distance of the image.  In this equation, the dot product $\myvec{p} \cdot
\myvec{U}$ is the position of the pixel along the axis of the projection profile
in the image defined by $\phi$. The values of $N$ and $s$ in
\refeqn{classicproj} help map each pixel to a bin index in the range $0$ to $N$.
}



\begin{figure}[t]
\centering
\begin{center}
\subfigure[Generating the projection profile of the text from a vanishing point $\myvec{V}$.]{
	\begin{tabular}{c}
		% \epsfig{figure=images/ppfan.eps,width=70mm}\label{projproffing}
		\epsfig{figure=images/fan93imgH.eps,width=100mm}\label{projproffing}
	\end{tabular}
}
% \hspace*{5mm}
% \subfigure[The winning projection profile and a poor example]{
	% \begin{tabular}{c}
		% \epsfig{figure=images/prlrunning/bestpp.eps,width=45mm} \\
		% \epsfig{figure=images/prlrunning/badpp.eps,width=45mm}
		% \label{projprofs}
	% \end{tabular}
% }
% \epsfig{figure=images/prlrunning/fan93img.eps,width=40mm}
\subfigure[The winning projection profile from vanishing point $\myvec{V}_A$ and a poor example from $\myvec{V}_B$.]{
	\epsfig{figure=images/chem002projprofsH.eps,width=100mm}
	\label{projprofs}
}
\end{center}
% \vspace*{0mm}
% \caption{Two potential vanishing points $\myvec{V}_A$ and $\myvec{V}_B$, and their projection profiles.}
\caption{Generating projection profiles from \reffig{runbin}}
\label{projprofsfig}
\end{figure}




For a perspectively skewed target, instead of an angle $\phi$, we have the point of projection
$\myvec{V}$ on the image plane, which has two degrees of freedom. The bins,
rather than representing parallel slices of the image along a particular
direction, must represent angular slices projecting from $\myvec{V}$.  Hence, we
refine (\ref{classicproj}) to map from an image pixel $\myvec{p}$ into a bin
$B_i$ as follows:


\begin{equation} \label{persproj} i( \myvec{p,\myvec{V}} ) = \frac{1}{2} N + N \frac{ \mbox{ang}(\myvec{V},\myvec{V}-\myvec{p}) }{ \Delta \theta } \end{equation}

{ \parindent 0mm 
where $\mbox{ang}(\myvec{V},\myvec{V}-\myvec{p})$ is the angle
between pixel $\myvec{p}$ and the centre of the image, relative to the vanishing
point $\myvec{V}$, and $\Delta \theta$ is the size of the angular range within
which the text is contained, again relative to the vanishing point
$\myvec{V}$. $ \Delta \theta$ is obtained from 
}
%$ \Delta \theta =
%\mbox{ang}(\myvec{V}+\myvec{t},\myvec{V}-\myvec{t}) $

% \begin{equation} \Delta \theta = \mbox{ang}(\myvec{V}+\myvec{t},\myvec{V}-\myvec{t}) \end{equation}
\begin{equation} \Delta \theta = \mbox{ang}(\myvec{T}_{\mbox{\tiny L}},\myvec{T}_{\mbox{\tiny R}}) \end{equation}

{ \parindent 0mm 
% where $\myvec{t}$ is a vector perpendicular to $\myvec{V}$ with
% magnitude equal to the radius of the bounding circle of the text region
where $\myvec{T}_{\mbox{\tiny L}}$ and $\myvec{T}_{\mbox{\tiny R}}$ are the two
points on the bounding circle whose tangents pass through $\myvec{V}$
(shown
in \reffig{projproffing}).  Unlike $s$ in (\ref{classicproj}), it can be seen that
$\Delta \theta$ is dependent on the point of projection $\myvec{V}$.  In fact
$\Delta \theta \rightarrow 0$ as $\myvec{V}_r \rightarrow \infty$ since more
distant vanishing points view the text region through a smaller angular range.
% The use of $\myvec{t}$ to find $\Delta\theta$ ensures that the angular range
The use of $\myvec{T}_{\mbox{\tiny L}}$ and $\myvec{T}_{\mbox{\tiny R}}$ to find $\Delta\theta$ ensures that the angular range
over which the text region is being analysed is as closely focused on the text
as possible, without allowing any of the text pixels to fall outside the range
of the projection profile's bins.  This is vital in order for the generated
profiles to be comparable, and also beneficial computationally since no bins
need to be generated for the angular range $2 {\pi}-\Delta \theta$ which is
absent of text.
}

\begin{comment}
Relative to a vanishing point $\myvec{V}$, the text region under examination
will fall entirely within a range of angles, forming an arc or wedge shape
extending from $\myvec{V}$.  The two angles which form this enclosing wedge are
used as the left and right bounds of the projection profile.  Hence all pixels
in the text region will map correctly to a bin in the projection profile, whilst
ensuring that the text fully spans the projection and a useful profile is
obtained.  Without this resizing of the window over which to collect the
projection profile for each vanishing point, the profiles obtained will not
relate to each other.  This is because more distant vanishing points find the
text lying within a smaller angular range, which will produce a drastically
different projection unless we focus on the relevant range.
\end{comment}

% For example, if we were to use a large window of $0-2\pi$ to generate all the
%projection profiles, those from the more distant vanishing points would find the
%projection congregating in fewer and fewer bins! 

% We have experimented with taking the entropy, squared-sum, and
% derivative-squared-sum of the projection profile, where:

% \begin{equation} \mbox{Entropy }E(B) = \sum_{i=1}^{N}{B_{i} log(B_{i})} \end{equation}
% \begin{equation} S(B) = \sum_{i=1}^{N}{{B_{i}}^{2}} \end{equation}
% \begin{equation} \mbox{Derivative-squared-sum }DSS(B) = \sum_{i=1}^{N-1}{{(B_{i+1}-B_{i}})^{2}} \end{equation}

% We found $SS(B)$ and $DSS(B)$ always respond accurately
% on a text only region, but in the presence of noisy data the two measures
% perform differently.
% % The entropy measure $E(B)$ performed significantly worse than the other two.
% In our experiments, and in \reffig{ppmap}, the measure we used was the squared-sum, $SS(B)$.
% % The introduction of different types of noise produced different
% % results from each, but we shall not go into that here.
% % although the derivatives can be more informative in the presence of
% % types of noise.
% We note that the accurate calculation of derivatives requires
% the generation of a high-resolution projection profile with proper modelling of the
% pixel contributions.
% Since the sum of squares does not require such accurate data, for this measure
% projection profiles may be generated far more computationally efficiently
% with a lower resolution of bins and a simplistic model
% of pixel contribution.

Having accumulated projection profiles for all the hypothesised vanishing points
using (\ref{persproj}), a simple measure of confidence is
applied to each projection profile $B$ to mark its significance.
The confidence measure must be chosen to respond favourably to projection profiles
with distinct peaks and troughs.
Since straight lines are most clearly distinguishable from the point where they
intersect, this horizontal vanishing point and its neighbourhood should be
favoured by the measure. We experimented with taking the entropy, squared-sum, and
derivative-squared-sum of the projection profiles, where:
\begin{equation} \mbox{Entropy: } \hspace*{5mm} E(B) = \sum_{i=1}^{N}{B_{i} log(B_{i})} \end{equation}
\begin{equation} \mbox{Squared-sum: } \hspace*{5mm} S(B) = \sum_{i=1}^{N}{{B_{i}}^{2}} \end{equation}
\begin{equation} \mbox{Derivative-squared-sum: } \hspace*{5mm} \nabla S(B) = \sum_{i=1}^{N-1}{{(B_{i+1}-B_{i}})^{2}} \end{equation}

The derivative measure $\nabla S(B)$ proved far more resilient to noise than the
other measures, which were easily mislead to view narrow paragraphs from the
top or bottom, rather than the side.
This made $\nabla S(B)$ the only viable measure for the hierarchical scan or noisy
images.

\begin{comment}
The entropy measure $E(B)$ performed significantly worse in all the above
measures.  $S(B)$ and $S_d(B)$ always responded well with the $S(B)$ measure
deteriorating rapidly as the image became noisier. CHECK JOEY.  $S_d(B)$
responded best even in presence of noise and is also more efficient to compute.
The confidence of each of the vanishing points with regard to the binarised text
in \reffig{runbin} is plotted in \reffig{ppmap}, where darker pixels represent a
larger squared-sum, and a more likely vanishing point.
\end{comment}


\begin{figure}[t!]
\centering
\begin{center}
\subfigure[Fully justified]{\epsfig{figure=images/simulated-full.ps,height=42mm}\label{simulated-full}} 
% \subfigure[Left justified]{\epsfig{figure=images/simulated-left.ps,width=41mm}\label{simulated-left}} 
\hspace*{3mm}
\subfigure[Centrally justified]{\epsfig{figure=images/simulated-cen.ps,height=42mm}\label{simulated-left}} 
\hspace*{3mm}
\subfigure[Left justified with extraneous elements]
{\epsfig{figure=images/simulated-left-more.ps,height=42mm}\label{simulated-left-more}} 
\end{center}
\vspace*{0mm}
\caption{Examples of simulated images used for performance analysis.}
\label{simimages}
\end{figure}

To locate the vanishing point accurately, the resolution of the
search space must be sufficient to hypothesise a large number of potential
vanishing points.  During experiments we found empirically that $10^4$
vanishing points was reasonable. Since each vanishing point examined requires
the generation and analysis of a projection profile, a full search of the
space, as shown in \reffig{ppmap}, is computationally expensive.  However, due
to the large scale features of the search space, we introduced an efficient
hierarchical approach to the search.  An initial scan of the search space at a
low resolution is performed, requiring the generation of only a few hundred
projection profiles.  Adaptive thresholding is then applied to the confidence
measures of these projection profiles, to extract the most interesting regions
of the search space.  (In our experiments, this was taken to be the top scoring
2\% of the space.) A full resolution scan is then performed on these
interesting regions in the search space, requiring the generation of a few
further projection profiles close to the expected solution.  Finally, the
projection profile with the largest confidence is chosen as the horizontal
vanishing point of the text plane.  
This hierarchical search reduced the processing time on an HP-UX from over 40
seconds to about 3 seconds. In all our experiments, this adapted search found
the same horizontal vanishing point as when we performed a full search.
The winning projection profile and an
example of a poor projection profile are shown in \reffig{projprofs}, and
marked in \reffig{ppmap} with a white cross and a black cross respectively.
% The hierarchical search reduces the processing time on a Sun Enterprise from
% over two minutes to under ten seconds.

% It is worth noting that for this adapted algorithm, again the derivative measure
% $S_d(B)$ performs far more efficiently than the sqaured-sum and entropy
% measures, in that they can mislead the hierarchical search by also responding
% favourably to the {\em vertical} vanishing point of a `thin' document, i.e. a
% document which has been rotated about its vertical axes.

% JOEY- Can we say something more about noise similar to this commented 
% paragraph I found in your tex file? basically we need to state a little more
% clearly about the reaction of our confidence measure to noise.
% Whilst different types of noise can create different problems,
% in general noisy pixels affect all of the projection profiles similarly,
% and the horizontal vanishing point still retains the greatest confidence.
% Despite general image noise and the resolution of the search space, this method 
% has consistently provided a good estimate of the 
% horizontal vanishing point in our experiments.
% Note that if the best projection profile is found on the edge of the unit
% circle, then the vanishing point is effectively at an infinite distance, and the
% horizontal axis of the text plane is parallel with the image plane.

In order to assess the performance of the algorithm, simulated images such as
those in \reffig{simimages} were generated at various orientations ranging from
$0^\circ$ to $90^\circ$ in both yaw and pitch, resulting in 900 test images. 
\reffig{hvpaccuracy} shows the accuracy of recovery of the horizontal vanishing
point (HVP) for these images, 
calculated as the relative distance of the located vanishing point $\myvec{H}$ 
from the ground truth $\myvec{H}_{\mbox{\small gt}}$, given by
\begin{equation} 
\mbox{HVP accuracy =} -\frac{|\myvec{H}-\myvec{H}_{\mbox{\small gt}}|}{|\myvec{H}_{\mbox{\small gt}}|}
\end{equation}
As can be seen, the proposed method achieves excellent accuracy across a wide
range of orientations. However, not unexpectedly, the performance begins to
drop as the orientation of the plane approaches $90^\circ$ in yaw or pitch. In
these cases, the document has been rotated so as to be almost orthogonal to the
view plane, and hence nearly invisible in the image, explaining the reduction
in performance. The slope of the graph at low yaw may be attributed to the
discretisation of the search space $C$. Since the vanishing points in these
situations lie close to infinity, the distances of the located vanishing points
can not be precise. Nevertheless, the vanishing point chosen will be in the
correct direction, and suitably large so as not to affect further processing.


\begin{figure}[h]
\centering
\begin{center}
% \epsfig{figure=images/accuracy-hvp-jagged,width=60mm,angle=-90}
% \end{center}
% \vspace*{0mm}
% \caption{Accuracy of recovery of the horizontal vanishing point for simulated
% paragraphs at various orientations. SHOULD COMBINE WITH NEXT FIGURE}
% \label{hvpaccuracy}
% \end{figure}
\subfigure[
	Accuracy of recovery of a simulated paragraph at various orientations.
]{
	\epsfig{figure=images/accuracy-hvp-jagged.eps,width=42mm,angle=90}
	\label{hvpaccuracy}
} 
\hspace*{4mm}
\subfigure[
	Accuracy averaged over all orientations plotted against
	increasing image noise.
]{
	\epsfig{figure=images/accuracy-against-noise-on-full-and-hier.eps,height=42mm}
	\label{noiseFullVHier}
} 
% \begin{figure}[t]
% \centering
% \begin{center}
% \epsfig{figure=images/accuracy-against-noise-on-full-and-hier.eps,width=80mm}
\end{center}
% \vspace*{0mm}
% \caption{Average accuracy of recovery of the horizontal vanishing point for
% simulated paragraphs against increasing image noise.}
% \label{noiseFullVHier}
\caption{Accuracy of recovery of the horizontal vanishing point.}
\end{figure}




To examine the influence of noise on the performance, we ran the same test
repeatedly on images with increasing proportions of speckled noise, and took
the average of the results over all different orientations.
\reffig{noiseFullVHier} shows the average accuracy of recovery of the
horizontal vanishing point against image noise.  The results when using a full
scan of the search space are also plotted, to demonstrate that there is no loss
of accuracy when using the hierarchical scan.  A numerical analysis of the
performance is given in \reftab{accuracytable} and discussed at the end of this
paper.



