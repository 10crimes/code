\documentclass[a4paper]{article}

\setlength{\parindent}{0pt}
\setlength{\parskip}{2mm}

\begin{document}

%% My favourite is number 2, currently uncommented.
% \title{On Document Recovery from Real Scene Images}
 \title{On the Recovery of Oriented Documents from Single Images}
% \title{On the Recovery of Perspectively Skewed Documents from Single Images}
%\title{Robust recovery of Perspectively Skewed text}
\date{}
% \date{\today}
\author{}
\maketitle


In many workplaces both electronic and paper documents are read, edited and
transferred. A point-and-click document capture system, avoiding the cumbersome
use of a fixed desktop scanner is highly beneficial. The major problem is
firstly to locate the text and then to correctly recover the perspective view of
it, ready for OCR. We have dealt with the extraction of blocks of large text or
paragraphs in images of real scenes [1] but only partially dealt with their
perspective recovery [2,3].  This paper addresses this as-yet unsolved problem: the
recovery of left, right or centrally justified paragraphs in a single
image. Other benefits of full document/text capture and recovery include a
translation tool for tourists and as an aid for the visually impaired, amongst
others.

Work related to perspective recovery has mainly been based on planar texture
analysis [10,5]. For example, [10] found the affine distortion
in the image power spectra along straight lines to determine first the
orientation of the vanishing line and then its position. Most works on OCR-based
document capture assume that the text is already
fronto-parallel.  Other than our recent work in [3], the only other (soon to
be published) work known to the authors on text-based perspective recovery is [9].  The author seeks visual clues in the image which correspond to
horizontal and vertical features on the document plane.  However, the vertical
vanishing point of the text plane cannot be robustly estimated when only one
vertical clue is present.  Examples of this situation arise when the document is
single-column or is not fully justified.  We present a novel approach to recover
both horizontal and vertical vanishing points that can cope with all these
situations, providing comparative experimental results as well as a full
performance evaluation. We discuss limitations of the solution and possible
improvements.


Key words:
Document recognition, Shape, Surface Geometry, Perspective Recovery, Planar Orientation, Vanishing Points, Document Structure


% \bibliographystyle{plain}
% \bibliography{jrefs}



\end{document}

