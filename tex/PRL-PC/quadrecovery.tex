\section{Recovery of fronto-parallel view}

% Given a partially calibrated camera for which only the focal lengthis unknown, a quadrilateral in the image which is known to correspond to some rectangle in the world is sufficient to determine the focal length.

In some applications we may not know the focal length of the camera used to capture the image.  However a quadrilateral in the image which is known to correspond to a rectangle in the world is sufficient to recovert the focal length of the camera.  We shall briefly demonstrate ...

The lines joining the focal point $O$ to the two vanishing points $\myvec{HVP}$ and $\myvec{VVP}$ in the image plane are parallel to the horizontal and vertical vectors of the document.  Since we know the edges of the rectangular document are at right-angles, we can write:

\begin{equation}
( \myvec{HVP}_x , \myvec{HVP}_y , f ) \cdot ( \myvec{VVP}_x , \myvec{VVP}_y , f ) = 0
\end{equation}

This constraint expands to:

\begin{equation}
f = \sqrt{ - \myvec{HVP}_x \myvec{VVP}_x - \myvec{HVP}_y \myvec{VVP}_y }
\end{equation}

% \begin{equation}
% ( \myvec{HVP}_x , \myvec{HVP}_y ) \cdot ( \myvec{VVP}_x , \myvec{VVP}_y ) > 0 ~.
% \end{equation}

It is worth noting that no solution exists for $f$ if $ ( \myvec{HVP}_x , \myvec{HVP}_y ) \cdot ( \myvec{VVP}_x , \myvec{VVP}_y ) > 0 $.  This situation, where the angle between the two vanishing points on the image plane is acute, corresponds to a quadrilateral which cannot possibly be a rectangle in the scene.  If such a quadrilateral is encountered during processing, we could hypothesise that the document is in fact slanted on the text plane, or that the quadrilateral does not correspond to a document in the scene, and should be ignored.

Having obtained the focal length of the camera, we may now recover a fronto-parallel view of the document.  The interpolation takes place in world space rather than image space.
